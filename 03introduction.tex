
\section{INTRODUCTION} \label{introduction}

 Technological advances such as the broadband internet, powerful computers, mobile devices with exceptional cameras, improved and intuitive design software as well as widely available online tutorials have lowered the barrier and most people with basic technological  literacy can become creators of digital work. Every single day more than 1.1 trillion digital files such as audio, photographs, videos, documents, ebooks and design files are created. 90\% of world's data was created in the last two years alone \cite{ibm}. 

The Web is the ideal medium that allows creators to reach a worldwide audience and sell their work while being able to sustain themselves. In practice though, this is far from reality. Current channels for content licensing and monetization fail to provide creators the freedom to decide how to license and distribute their work and usually, the monetary compensation for work sold is severely reduced due to large fees applied on sales.

At BlockLabs we believe that blockchain and specifically the Ethereum \cite{buterin2013ethereum} Blockchain and Virtual Machine offer the ideal medium on which a system for fair content licensing and monetization can be built. The Ethereum blockchain is public, distributed and immutable and enables the creation of arbitrary paying systems and structures while maintaining overall security. Also, it enables very fast transaction processing times, it has low fees and it is borderless. The Ethereum blockchain and virtual machine, provide the building blocks for an ecosystem of tools and services that can revolutionize the way digital content is licensed, distributed and sold.

BlockLicense is a project that aims to provide creators with all the tools that can make licensing an integral part of their creative workflow and provide buyers with several avenues to discover and license fairly-traded digital content. The BlockLicense Ecosystem is designed from the ground up with Licensing at its very core and can accommodate a wide range of licensing scenarios such as ones where only one creator is involved, to more complex ones with multiple stakeholders. 